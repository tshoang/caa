% \iffalse meta-comment
% 
% caa.ins
% 
% Copyright (C) 2020 University of Southampton
% 
% --------------------------------------------------------------------
% 
% This file may be distributed and/or modified under the
% conditions of the LaTeX Project Public License, either version 1.3c
% of this license or (at your option) any later version.
% The latest version of this license is in:
% 
%      http://www.latex-project.org/lppl.txt
% 
% and version 1.3c or later is part of all distributions of LaTeX 
% version 2008/05/04 or later.
% 
% This work has the LPPL maintenance status "author-maintained".
% 
% The Current Maintainer of this work is T.S. Hoang
% (<T dot S dot Hoang at ecs dot soton dot ac dot uk>).
%
% This work consists of the files caa.dtx, caa.ins,
% the derived file caa.sty, the generated documentation
% caa.pdf, and some sample requirements documents.
% 
% \fi
% 
% \iffalse
%<caa>\NeedsTeXFormat{LaTeX2e}\relax
%<caa>\ProvidesPackage{caa}
%<caa>    [2020/03/29 v0.1 Package for typesetting computer-aided assessments]
% 
%<*driver> 
\documentclass[a4paper]{ltxdoc}
\usepackage[answers, colour]{caa}
\usepackage{sample-bank}
\usepackage{nomencl}
\renewcommand{\nomname}{List of Questions}
\makenomenclature
\EnableCrossrefs
\CodelineIndex
\PageIndex
\RecordChanges

\begin{document}
\DocInput{caa.dtx}
\end{document}
%</driver>
% \fi
% 
% \CheckSum{127}
% 
% \CharacterTable
% {Upper-case    \A\B\C\D\E\F\G\H\I\J\K\L\M\N\O\P\Q\R\S\T\U\V\W\X\Y\Z
% Lower-case    \a\b\c\d\e\f\g\h\i\j\k\l\m\n\o\p\q\r\s\t\u\v\w\x\y\z
% Digits        \0\1\2\3\4\5\6\7\8\9
% Exclamation   \!     Double quote  \"     Hash (number) \#
% Dollar        \$     Percent       \%     Ampersand     \&
% Acute accent  \'     Left paren    \(     Right paren   \)
% Asterisk      \*     Plus          \+     Comma         \,
% Minus         \-     Point         \.     Solidus       \/
% Colon         \:     Semicolon     \;     Less than     \<
% Equals        \=     Greater than  \>     Question mark \?
% Commercial at \@     Left bracket  \[     Backslash     \\
% Right bracket \]     Circumflex    \^     Underscore    \_
% Grave accent  \`     Left brace    \{     Vertical bar  \|
% Right brace   \}     Tilde         \~}
% 
% 
% \changes{v0.1}{2020/03/29}{Initial version}
% 
% \GetFileInfo{caa.sty}
% 
% \DoNotIndex{\\}
% \DoNotIndex{\DeclareOption}
% \DoNotIndex{\ProcessOptions}
% \DoNotIndex{\RequirePackage}
% \DoNotIndex{\arabic}
% \DoNotIndex{\begin}
% \DoNotIndex{\csname,\csuse}
% \DoNotIndex{\def,\do,\dolistloop}
% \DoNotIndex{\end,\endcsname,\expandafter}
% \DoNotIndex{\hline}
% \DoNotIndex{\ifstrequal,\iftoggle,\item}
% \DoNotIndex{\label,\labelformat,\listadd}
% \DoNotIndex{\medskip}
% \DoNotIndex{\newcommand,\newcounter,\newenvironment,\newtoggle,\nomenclature}
% \DoNotIndex{\quad}
% \DoNotIndex{\renewcommand,\renewenvironment,\ref,\refstepcounter}
% \DoNotIndex{\setcounter,\small}
% \DoNotIndex{\textsf,\textwidth,\togglefalse,\toggletrue}
% \DoNotIndex{\usepackage}
% \DoNotIndex{\value,\verb}
% \DoNotIndex{\xspace}
%
% \title{The \textsf{caa} package\thanks{This document
% corresponds to \textsf{caa}~\fileversion, dated~\filedate.}}
% \author{Thai Son Hoang\\ ECS, University of Southampton \\
% \texttt{<T dot S dot Hoang at ecs dot
% soton dot ac dot uk>}}
% \date{March 29, 2020}
% 
% \maketitle
% 
% ^^A %%%%% Abstract %%%%%
% \begin{abstract}
%   This package provides macros for typesetting computer-aided
%   assessments.  It was developed at the University of Southampton.
% \end{abstract}
% 
% ^^A %%%%% Table of contents %%%%%
% \tableofcontents
% 
% ^^A %%%%% Introduction %%%%%
% \section{Introduction}
% 
% This package was developed in order to ease the typesetting of
% computer-aided assessments in \LaTeX{}.
% 
% ^^A %%%%% Usage %%%%%%
% \section{Usage}
% \label{sec:usage}
% 
% Just like any other package, you need to request this package with a
% \verb|\usepackage| command in the preamble.  So in the simpler case
% (i.e., without any options), one just types
% 
% \iffalse ^^A BEGIN Produce comments only in the resulting style file
%<samplequestions>\documentclass[a4paper, 10pt]{article}
% \fi ^^A END Produce comments only in the resulting style file
% \iffalse ^^A BEGIN Ensure the macrocode to the resulting style file
%<*samplequestions> 
% \fi
%    \begin{macrocode}
\usepackage{caa}
%    \end{macrocode}
% \iffalse ^^A END Ensure the macrocode to the resulting style file
%</samplequestions> 
% \fi
% \iffalse ^^A BEGIN Produce comments only in the resulting style file
%<samplequestions>\title{Sample Document for CAA package\\(questions)}
%<samplequestions>\author{%
%<samplequestions>  Thai Son Hoang\\
%<samplequestions>  ECS, University of Southampton \\
%<samplequestions>  \texttt{<T dot S dot Hoang at ecs dot soton dot ac dot uk>}
%<samplequestions>}
%<samplequestions>\date{March 29, 2020}
%<samplequestions>
%<samplequestions>%%
%% This is file `sample-main.tex',
%% generated with the docstrip utility.
%%
%% The original source files were:
%%
%% caa.dtx  (with options: `samplemain')
%% 
%% This is a generated file.
%% 
%% Copyright (C) 2020 University of Southampton
%% 
%% This file may be distributed and/or modified under the
%% conditions of the LaTeX Project Public License, either version 1.3c
%% of this license or (at your option) any later version.
%% The latest version of this license is in:
%% 
%%    http://www.latex-project.org/lppl.txt
%% 
%% and version 1.3c or later is part of all distributions of LaTeX
%% version 2008/05/04 or later.
%% 
%% This work has the LPPL maintenance status "author-maintained".
%% 
%% The Current Maintainer of this work is T. S. Hoang
%% (<T dot S dot Hoang at ecs dot soton dot ac dot uk>).
%% 
\usepackage{sample-bank}

\usepackage{nomencl}
\renewcommand{\nomname}{List of Questions}
\makenomenclature

\begin{document}
\maketitle

\begin{table}[!htbp]
  \centering
  \begin{tabular}{|l|l|}
    \hline
    Category & Stats \\
    \hline
    \emph{Easy} & \printcategory{Easy} \\
    \emph{Medium} & \printcategory{Medium} \\
    \emph{Hard} & \printcategory{Hard} \\
    \hline
    \emph{\textbf{Total}} & \printcategory{Total} \\
    \hline
  \end{tabular}
  \caption{Summary}
  \label{tab:summary}
\end{table}

\SampleQuestion1

\SampleQuestion2

\SampleQuestion3

\SampleQuestion4


\printnomenclature

\end{document}
\endinput
%%
%% End of file `sample-main.tex'.

% \fi ^^A END Produce comments only in the resulting style file
%
% \noindent to load the package.
% 
% \subsection{Package Options}
% \label{sec:package-options-usage}
%
% \paragraph{Answers.}
% Loading package with the \verb|answers| option will enable the
% answer and explanation for the questions.
%
% \iffalse ^^A BEGIN Produce comments only in the resulting style file
%<sampleanswers>\documentclass[a4paper, 10pt]{article}
% \fi ^^A END Produce comments only in the resulting style file
% \iffalse ^^A BEGIN Ensure the macrocode to the resulting style file
%<*sampleanswers> 
% \fi
%    \begin{macrocode}
\usepackage[answers]{caa}
%    \end{macrocode}
% \iffalse ^^A END Ensure the macrocode to the resulting style file
%</sampleanswers> 
% \fi
% \iffalse ^^A BEGIN Produce comments only in the resulting style file
%<sampleanswers>\title{Sample Document for CAA packag\\(with answers)}
%<sampleanswers>\author{%
%<sampleanswers>  Thai Son Hoang\\
%<sampleanswers>  ECS, University of Southampton \\
%<sampleanswers>  \texttt{<T dot S dot Hoang at ecs dot soton dot ac dot uk>}
%<sampleanswers>}
%<sampleanswers>\date{March 29, 2020}
%<sampleanswers>
%<sampleanswers>%%
%% This is file `sample-main.tex',
%% generated with the docstrip utility.
%%
%% The original source files were:
%%
%% caa.dtx  (with options: `samplemain')
%% 
%% This is a generated file.
%% 
%% Copyright (C) 2020 University of Southampton
%% 
%% This file may be distributed and/or modified under the
%% conditions of the LaTeX Project Public License, either version 1.3c
%% of this license or (at your option) any later version.
%% The latest version of this license is in:
%% 
%%    http://www.latex-project.org/lppl.txt
%% 
%% and version 1.3c or later is part of all distributions of LaTeX
%% version 2008/05/04 or later.
%% 
%% This work has the LPPL maintenance status "author-maintained".
%% 
%% The Current Maintainer of this work is T. S. Hoang
%% (<T dot S dot Hoang at ecs dot soton dot ac dot uk>).
%% 
\usepackage{sample-bank}

\usepackage{nomencl}
\renewcommand{\nomname}{List of Questions}
\makenomenclature

\begin{document}
\maketitle

\begin{table}[!htbp]
  \centering
  \begin{tabular}{|l|l|}
    \hline
    Category & Stats \\
    \hline
    \emph{Easy} & \printcategory{Easy} \\
    \emph{Medium} & \printcategory{Medium} \\
    \emph{Hard} & \printcategory{Hard} \\
    \hline
    \emph{\textbf{Total}} & \printcategory{Total} \\
    \hline
  \end{tabular}
  \caption{Summary}
  \label{tab:summary}
\end{table}

\SampleQuestion1

\SampleQuestion2

\SampleQuestion3

\SampleQuestion4


\printnomenclature

\end{document}
\endinput
%%
%% End of file `sample-main.tex'.

% \fi ^^A END Produce comments only in the resulting style file
%
% \paragraph{Colouring Option.}
% Loading package with the \verb|colour| or \verb|color| options will enable the
% colour for indication of answers.
%
% \iffalse ^^A BEGIN Produce comments only in the resulting style file
%<sampleanswerscolour>\documentclass[a4paper, 10pt]{article}
% \fi ^^A END Produce comments only in the resulting style file
% \iffalse ^^A BEGIN Ensure the macrocode to the resulting style file
%<*sampleanswerscolour> 
% \fi
%    \begin{macrocode}
\usepackage[answers, colour]{caa}
%    \end{macrocode}
% \iffalse ^^A END Ensure the macrocode to the resulting style file
%</sampleanswerscolour> 
% \fi
% \iffalse ^^A BEGIN Produce comments only in the resulting style file
%<sampleanswerscolour>\title{Sample Document for CAA package\\
%<sampleanswerscolour>  (with answers and colour)}
%<sampleanswerscolour>\author{%
%<sampleanswerscolour>  Thai Son Hoang\\
%<sampleanswerscolour>  ECS, University of Southampton \\
%<sampleanswerscolour>  \texttt{<T dot S dot Hoang at ecs dot soton dot ac dot uk>}
%<sampleanswerscolour>}
%<sampleanswerscolour>\date{March 29, 2020}
%<sampleanswerscolour>
%<sampleanswerscolour>%%
%% This is file `sample-main.tex',
%% generated with the docstrip utility.
%%
%% The original source files were:
%%
%% caa.dtx  (with options: `samplemain')
%% 
%% This is a generated file.
%% 
%% Copyright (C) 2020 University of Southampton
%% 
%% This file may be distributed and/or modified under the
%% conditions of the LaTeX Project Public License, either version 1.3c
%% of this license or (at your option) any later version.
%% The latest version of this license is in:
%% 
%%    http://www.latex-project.org/lppl.txt
%% 
%% and version 1.3c or later is part of all distributions of LaTeX
%% version 2008/05/04 or later.
%% 
%% This work has the LPPL maintenance status "author-maintained".
%% 
%% The Current Maintainer of this work is T. S. Hoang
%% (<T dot S dot Hoang at ecs dot soton dot ac dot uk>).
%% 
\usepackage{sample-bank}

\usepackage{nomencl}
\renewcommand{\nomname}{List of Questions}
\makenomenclature

\begin{document}
\maketitle

\begin{table}[!htbp]
  \centering
  \begin{tabular}{|l|l|}
    \hline
    Category & Stats \\
    \hline
    \emph{Easy} & \printcategory{Easy} \\
    \emph{Medium} & \printcategory{Medium} \\
    \emph{Hard} & \printcategory{Hard} \\
    \hline
    \emph{\textbf{Total}} & \printcategory{Total} \\
    \hline
  \end{tabular}
  \caption{Summary}
  \label{tab:summary}
\end{table}

\SampleQuestion1

\SampleQuestion2

\SampleQuestion3

\SampleQuestion4


\printnomenclature

\end{document}
\endinput
%%
%% End of file `sample-main.tex'.

% \fi ^^A END Produce comments only in the resulting style file
%
% \subsection{Typesetting a Question Bank.}
% Using the |caa| package, ones can develop a question bank, and
% different exams can be drawn from the question bank. Typically, we
% use a style file for this purpose. For example, the following
% |sample-bank.sty| files contains some questions multiple choice
% questions.
%
% \iffalse ^^A BEGIN Ensure the macrocode to the resulting style file
%<*samplebank> 
% \fi
%    \begin{macrocode}
\NeedsTeXFormat{LaTeX2e}\relax
\ProvidesPackage{sample-bank}
    [2020/03/29 v0.1 Package for sample question banks]

\RequirePackage{caa}

\num\newcommand{\SampleQuestion1}{%
  \newquestion{Question01}{1}{%
    This is an easy question.

    \begin{multiplechoice}
      \distractor{A distractor 1}

      \distractor{Another distractor 2}

      \keyed{A keyed answer}

      \distractor{Another distractor 3}
    \end{multiplechoice}
  }
}

\num\newcommand{\SampleQuestion2}{%
  \newquestion[Medium]{Question02}{2}{%
    This is a medium question.

    \begin{multiplechoice}
      \dropdown{Entry 1}{\correct{option1}, \incorrect{option2},
        \incorrect{option3}, \incorrect{option4}}

      \dropdown{Entry 2}{\incorrect{option1}, \incorrect{option2},
        \incorrect{option3}, \correct{option4}}

      \dropdown{Entry 3}{\incorrect{option1}, \correct{option2},
        \incorrect{option3}, \incorrect{option4}}

      \dropdown{Entry 4}{\incorrect{option1}, \incorrect{option2},
        \correct{option3}, \incorrect{option4}}
    \end{multiplechoice}
  }
}

\num\newcommand{\SampleQuestion3}{%
  \newquestion[Easy]{Question03}{1}{%
    This is a easy question.

    \begin{multiplechoice}
      \dropdown{Entry 1}{\correct{option1}, \incorrect{option2},
        \incorrect{option3}, \incorrect{option4}}

      \dropdown{Entry 2}{\incorrect{option1}, \incorrect{option2},
        \incorrect{option3}, \correct{option4}}

      \dropdown{Entry 3}{\incorrect{option1}, \correct{option2},
        \incorrect{option3}, \incorrect{option4}}

      \dropdown{Entry 4}{\incorrect{option1}, \incorrect{option2},
        \correct{option3}, \incorrect{option4}}
    \end{multiplechoice}
  }
}

\num\newcommand{\SampleQuestion4}{%
  \newquestion[Hard]{Question04}{4}{%
    This is a hard question.

    \begin{multiplechoice}
      \dropdown{Entry 1}{\correct{option1}, \incorrect{option2},
        \incorrect{option3}, \incorrect{option4}}

      \dropdown{Entry 2}{\incorrect{option1}, \incorrect{option2},
        \incorrect{option3}, \correct{option4}}

      \dropdown{Entry 3}{\incorrect{option1}, \correct{option2},
        \incorrect{option3}, \incorrect{option4}}

      \dropdown{Entry 4}{\incorrect{option1}, \incorrect{option2},
        \correct{option3}, \incorrect{option4}}
    \end{multiplechoice}
  }
}
%    \end{macrocode}
% \iffalse ^^A END Ensure the macrocode to the resulting style file
%</samplebank> 
% \fi
%
% \subsection{Typesetting an Exam Using the Question Bank}
% An exam can be typeset by selecting the questions from the question
% bank by import the |sample-bank| package.
%
% \iffalse ^^A BEGIN Ensure the macrocode to the resulting style file
%<*samplemain> 
% \fi
%    \begin{macrocode}
\usepackage{sample-bank}

%    \end{macrocode}
% \iffalse ^^A END Ensure the macrocode to the resulting style file
%</samplemain> 
% \fi
%
% To produce the list of questions, we use the |nomencl| package.
% \iffalse ^^A BEGIN Ensure the macrocode to the resulting style file
%<*samplemain> 
% \fi
%    \begin{macrocode}
\usepackage{nomencl}
\renewcommand{\nomname}{List of Questions}
\makenomenclature

%    \end{macrocode}
% \iffalse ^^A END Ensure the macrocode to the resulting style file
%</samplemain> 
% \fi
%
% \iffalse ^^A BEGIN Produce comments only in the resulting style file
%<samplemain>\begin{document}
%<samplemain>\maketitle
%<samplemain>
% \fi ^^A END Produce comments only in the resulting style file
%
% A summary for the set of question can by type set using
% |printcategory| to print the number of questions and the number of
% mark in each category. The following
% \iffalse ^^A BEGIN Ensure the macrocode to the resulting style file
%<*samplemain> 
% \fi
%    \begin{macrocode}
\begin{table}[!htbp]
  \centering
  \begin{tabular}{|l|l|}
    \hline
    Category & Stats \\
    \hline
    \emph{Easy} & \printcategory{Easy} \\
    \emph{Medium} & \printcategory{Medium} \\
    \emph{Hard} & \printcategory{Hard} \\
    \hline
    \emph{\textbf{Total}} & \printcategory{Total} \\
    \hline
  \end{tabular}
  \caption{Summary}
  \label{tab:summary}
\end{table}

%    \end{macrocode}
% \iffalse ^^A END Ensure the macrocode to the resulting style file
%</samplemain> 
% \fi
% will produce something like this.
% {
% \DeleteShortVerb{\|}
% \begin{table}[!htbp]
%   \centering
%   \begin{tabular}{|l|l|}
%     \hline
%     Category & Stats \\
%     \hline
%     \emph{Easy} & \printcategory{Easy} \\
%     \emph{Medium} & \printcategory{Medium} \\
%     \emph{Hard} & \printcategory{Hard} \\
%     \hline
%     \emph{\textbf{Total}} & \printcategory{Total} \\
%     \hline
%   \end{tabular}
%   \caption{Summary}
%   \label{tab:summary}
% \end{table}
% \MakeShortVerb{\|}
% }
%
% The question can be typeset by using the question macros defined
% earlier in the question bank. For example, the following
% \iffalse ^^A BEGIN Ensure the macrocode to the resulting style file
%<*samplemain> 
% \fi
%    \begin{macrocode}
\SampleQuestion1

%    \end{macrocode}
% \iffalse ^^A END Ensure the macrocode to the resulting style file
%</samplemain> 
% \fi
% will produce (with |answers| and |colour| options enabled)
% \SampleQuestion1
%
% Similarly, the following
% \iffalse ^^A BEGIN Ensure the macrocode to the resulting style file
%<*samplemain> 
% \fi
%    \begin{macrocode}
\SampleQuestion2

%    \end{macrocode}
% \iffalse ^^A END Ensure the macrocode to the resulting style file
%</samplemain> 
% \fi
% will produce (with |answers| and |colour| options enabled)
% \SampleQuestion2
% Furthermore,
% \iffalse ^^A BEGIN Ensure the macrocode to the resulting style file
%<*samplemain> 
% \fi
%    \begin{macrocode}
\SampleQuestion3

%    \end{macrocode}
% \iffalse ^^A END Ensure the macrocode to the resulting style file
%</samplemain> 
% \fi
% will produce (with |answers| and |colour| options enabled)
% \SampleQuestion3
% and, finally
% \iffalse ^^A BEGIN Ensure the macrocode to the resulting style file
%<*samplemain> 
% \fi
%    \begin{macrocode}
\SampleQuestion4

%    \end{macrocode}
% \iffalse ^^A END Ensure the macrocode to the resulting style file
%</samplemain> 
% \fi
% will produce (with |answers| and |colour| options enabled)
% \SampleQuestion4
%
% The list of questions can be produced using the following command
% from the |nomencl| package. The following
% \iffalse ^^A BEGIN Ensure the macrocode to the resulting style file
%<*samplemain> 
% \fi
%    \begin{macrocode}

\printnomenclature

%    \end{macrocode}
% \iffalse ^^A END Ensure the macrocode to the resulting style file
%</samplemain> 
% \fi
% will produce
% \printnomenclature
%
% \iffalse ^^A BEGIN Produce comments only in the resulting style file
%<samplemain>\end{document}
% \fi ^^A END Produce comments only in the resulting style file
%
% \subsection{Compilation}
% \label{sec:compilation}
%
% The statistics for the each category require two compilation
% passes.  The list of questions requires a run of |makeindex|. So the
% following steps requires to compile an exam (assuming that we use
% |pdflatex|).
% \begin{enumerate}
% \item |pdflatex <exam>|
% \item |makeindex <exam>.nlo -s nomecl.ist -o <exam>.nls|
% \item |pdflatex <exam>|
% \item |pdflatex <exam>|
% \end{enumerate}
%
% \StopEventually{
% \PrintChanges
% \PrintIndex
% }
%
% ^^A %%%%% Implementation %%%%%
% \section{Implementation}
% \label{sec:implementation}
%
% ^^A %%% Package Loading %%% 
% \subsection{Package Dependencies}
% \label{sec:package-dependencies}
% \iffalse ^^A BEGIN Produce comments only in the resulting style file
%<caa>
%<caa>%%%%% BEGIN Package loading %%%%%
% \fi ^^A END Produce comments only in the resulting style file
%
% Our implementation uses |numdef| package for macros with numbering
% suffix.
%
% \iffalse ^^A BEGIN Produce comments only in the resulting style file
%<caa>
%<caa>% Package for commands with numbering suffix
% \fi ^^A END Produce comments only in the resulting style file
%
% \iffalse ^^A BEGIN Ensure the macrocode to the resulting style file
%<*caa> 
% \fi
%    \begin{macrocode}
\RequirePackage{numdef}
%    \end{macrocode}
% \iffalse ^^A END Ensure the macrocode to the resulting style file
%</caa> 
% \fi
% We use |nomencl| package to produce the list of the questions.
% \iffalse ^^A BEGIN Produce comments only in the resulting style file
%<caa>
%<caa>% Package for nomenclature, used to produce the list of questions
% \fi ^^A END Produce comments only in the resulting style file
% \iffalse ^^A BEGIN Ensure the macrocode to the resulting style file
%<*caa> 
% \fi
%    \begin{macrocode}
\RequirePackage{nomencl}
%    \end{macrocode}
% \iffalse ^^A END Ensure the macrocode to the resulting style file
%</caa> 
% \fi
% For colours, we use |xcolor| package.
% \iffalse ^^A BEGIN Produce comments only in the resulting style file
%<caa>
%<caa>% Package required for colours
% \fi ^^A END Produce comments only in the resulting style file
% \iffalse ^^A BEGIN Ensure the macrocode to the resulting style file
%<*caa> 
% \fi
%    \begin{macrocode}
\RequirePackage{xcolor}
%    \end{macrocode}
% \iffalse ^^A END Ensure the macrocode to the resulting style file
%</caa> 
% \fi
% We use |pifont| for checked mark and crossed mark.
% \iffalse ^^A BEGIN Produce comments only in the resulting style file
%<caa>
%<caa>% Package for PostScript standard Symbol and Dingbats fonts
% \fi ^^A END Produce comments only in the resulting style file
% \iffalse ^^A BEGIN Ensure the macrocode to the resulting style file
%<*caa> 
% \fi
%    \begin{macrocode}
\RequirePackage{pifont}
%    \end{macrocode}
% \iffalse ^^A END Ensure the macrocode to the resulting style file
%</caa> 
% \fi
% To store the total number of questions, we use |totcount| package.
% \iffalse ^^A BEGIN Produce comments only in the resulting style file
%<caa>
%<caa>% Package for total counter value
% \fi ^^A END Produce comments only in the resulting style file
% \iffalse ^^A BEGIN Ensure the macrocode to the resulting style file
%<*caa> 
% \fi
%    \begin{macrocode}
\RequirePackage{totcount}
%    \end{macrocode}
% \iffalse ^^A END Ensure the macrocode to the resulting style file
%</caa> 
% \fi
% We use |ifthen| package for conditional statements.
% \iffalse ^^A BEGIN Produce comments only in the resulting style file
%<caa>
%<caa>% LaTeX Conditional statements
% \fi ^^A END Produce comments only in the resulting style file
% \iffalse ^^A BEGIN Ensure the macrocode to the resulting style file
%<*caa> 
% \fi
%    \begin{macrocode}
\RequirePackage{ifthen}
%    \end{macrocode}
% \iffalse ^^A END Ensure the macrocode to the resulting style file
%</caa> 
% \fi
% Finally, we use |etoolbox| to get access to LaTeX front-end to e-TeX
% primitives.
% \iffalse ^^A BEGIN Produce comments only in the resulting style file
%<caa>
%<caa>% Utility package for LaTeX front-end to e-TeX primitives
% \fi ^^A END Produce comments only in the resulting style file
% \iffalse ^^A BEGIN Ensure the macrocode to the resulting style file
%<*caa> 
% \fi
%    \begin{macrocode}
\RequirePackage{etoolbox}
%    \end{macrocode}
% \iffalse ^^A END Ensure the macrocode to the resulting style file
%</caa> 
% \fi
% \iffalse ^^A BEGIN Produce comments only in the resulting style file
%<caa>%%%%% END Package loading %%%%%
%<caa>
% \fi ^^A END Produce comments only in the resulting style file
%
% \subsection{Package Options}
% \label{sec:package-options}
% 
% We define some options for typesetting the questions.
% \iffalse ^^A BEGIN Produce comments only in the resulting style file
%<caa>%%%%% BEGIN Declaration of options %%%%%
%<caa>% ========================
% \fi ^^A END Produce comments only in the resulting style file
%
% \subsubsection{Answers option}
% \label{sec:answers-option}
%
% We first declare some internal macros that can be updated when the
% |answers| package option is enabled.
%
% ^^A BEGIN \explanation macro
% \begin{macro}{\explanation}
% \changes{v0.1}{2020/03/29}{Initial version}
% Command |\explanation| is used to typeset the explanation to the
% answer.  By default, this is disabled, i.e., it gobbles the
% argument.  It is enabled when the package option |answers| is set.
% \iffalse ^^A BEGIN Produce comments only in the resulting style file
%<caa>
%<caa>% Macro to typeset the explaination. It is disabled by default.
%<caa>% It is enabled with the *answers* package option
%<caa>%
%<caa>% Arguments:
%<caa>% 1. The input explanation to be displayed when the *answers*
%<caa>% package option is enabled.
% \fi ^^A END Produce comments only in the resulting style file
% \iffalse ^^A BEGIN Ensure the macrocode to the resulting style file
%<*caa> 
% \fi
%    \begin{macrocode}
\newcommand{\explanation}[1]{}
%    \end{macrocode}
% \iffalse ^^A END Ensure the macrocode to the resulting style file
%</caa> 
% \fi
% \end{macro}
% ^^A END \explanation macro
%
% ^^A BEGIN Multiple choice environment
% \begin{environment}{multiplechoice}
%   \changes{v0.1}{2020/03/29}{Initial version}
%   We create a new environment for multiple choice question. It is
%   the same as |itemize|.
% \iffalse ^^A BEGIN Produce comments only in the resulting style file
%<caa>
%<caa>% Environment to typeset the multiple choice questions. It is
%<caa>% the same as *itemize* enviroment.
% \fi ^^A END Produce comments only in the resulting style file
% \iffalse ^^A BEGIN Ensure the macrocode to the resulting style file
%<*caa> 
% \fi
%    \begin{macrocode}
\newenvironment{multiplechoice}{\begin{itemize}}{\end{itemize}}
%    \end{macrocode}
% \iffalse ^^A END Ensure the macrocode to the resulting style file
%</caa> 
% \fi
% \end{environment}
% ^^A END Multiple choice environment
% 
% ^^A BEGIN \keyed macro
% \begin{macro}{\keyed}
% \changes{v0.1}{2020/03/29}{Initial version}
% Command |\keyed| is used to typeset keyed answers in multiple choice
% question. By default, it shows the the input argument.  When the
% |answers| package option is enabled, a checked mark is used to
% indicate it is the keyed answer.
% \iffalse ^^A BEGIN Produce comments only in the resulting style file
%<caa>
%<caa>% Macro to typeset a keyed answer for multiple choice question.
%<caa>%
%<caa>% Arguments:
%<caa>% 1. The input answer. When the *answers* package option is
%<caa>% enabled, a checked mark is used to indicate a keyed answer.
% \fi ^^A END Produce comments only in the resulting style file
% \iffalse ^^A BEGIN Ensure the macrocode to the resulting style file
%<*caa> 
% \fi
%    \begin{macrocode}
\newcommand{\keyed}[1]{\item #1}
%    \end{macrocode}
% \iffalse ^^A END Ensure the macrocode to the resulting style file
%</caa> 
% \fi
% \end{macro}
% ^^A END \keyed macro
%
% ^^A BEGIN \distractor macro
% \begin{macro}{\distractor}
% \changes{v0.1}{2020/03/29}{Initial version}
% Command |\distractor| is used to typeset distractor answers in
% multiple choice question. By default, it shows the the input
% argument.  When the |answers| package option is enabled, a crossed
% mark is used to indicate it is the distractor answer.
% \iffalse ^^A BEGIN Produce comments only in the resulting style file
%<caa>
%<caa>% Macro to typeset a distractor answer for multiple choice
%<caa>% question.
%<caa>%
%<caa>% Arguments:
%<caa>% 1. The input answer. When the *answers* package option is
%<caa>% enabled, a crossed mark is used to indicate a distractor
%<caa>% answer.
% \fi ^^A END Produce comments only in the resulting style file
% \iffalse ^^A BEGIN Ensure the macrocode to the resulting style file
%<*caa> 
% \fi
%    \begin{macrocode}
\newcommand{\distractor}[1]{\item #1}
%    \end{macrocode}
% \iffalse ^^A END Ensure the macrocode to the resulting style file
%</caa> 
% \fi
% \end{macro}
% ^^A END \distractor macro
%
% ^^A BEGIN \dropdown macro
% \begin{macro}{\dropdown}
% \changes{v0.1}{2020/03/29}{Initial version}
% Command |\dropdown| can be used to typeset dropdown answers in
% multiple choice question.
% \iffalse ^^A BEGIN Produce comments only in the resulting style file
%<caa>
%<caa>% Macro to typeset dropdown answers for multiple choice
%<caa>% question.
%<caa>%
%<caa>% Arguments:
%<caa>% 1. The question.
%<caa>% 2. The dropdown options.
% \fi ^^A END Produce comments only in the resulting style file
% \iffalse ^^A BEGIN Ensure the macrocode to the resulting style file
%<*caa> 
% \fi
%    \begin{macrocode}
\newcommand{\dropdown}[2]{%
  \item \emph{#1}: #2 
}
%    \end{macrocode}
% \iffalse ^^A END Ensure the macrocode to the resulting style file
%</caa> 
% \fi
% \end{macro}
% ^^A END \dropdown macro
%
% ^^A BEGIN \correct macro
% \begin{macro}{\correct}
% \changes{v0.1}{2020/03/29}{Initial version}
% Command |\correct| is used to typeset correct options in dropdown
% answers.  By default, it shows the the input argument.  When the
% |answers| package option is enabled, a checked mark is used to
% indicate it is the correct option.
% \iffalse ^^A BEGIN Produce comments only in the resulting style file
%<caa>
%<caa>% Macro to typeset correct options for dropdown answers
%<caa>%
%<caa>% Arguments:
%<caa>% 1. The input answer. When the *answers* package option is
%<caa>% enabled, a checked mark is used to indicate a correct option.
% \fi ^^A END Produce comments only in the resulting style file
% \iffalse ^^A BEGIN Ensure the macrocode to the resulting style file
%<*caa> 
% \fi
%    \begin{macrocode}
\newcommand{\correct}[1]{#1}
%    \end{macrocode}
% \iffalse ^^A END Ensure the macrocode to the resulting style file
%</caa> 
% \fi
% \end{macro}
% ^^A END \correct macro
%
% ^^A BEGIN \incorrect macro
% \begin{macro}{\incorrect}
% \changes{v0.1}{2020/03/29}{Initial version}
% Command |\incorrect| is used to typeset incorrect options in
% dropdown answers.  By default, it shows the the input argument.
% When the |answers| package option is enabled, a crossed mark is used
% to indicate it is the incorrect answer.
% \iffalse ^^A BEGIN Produce comments only in the resulting style file
%<caa>
%<caa>% Macro to typeset incorrect options for dropdown answers
%<caa>%
%<caa>% Arguments:
%<caa>% 1. The input answer. When the *answers* package option is
%<caa>% enabled, a crossed mark is used to indicate a incorrect
%<caa>% option.
% \fi ^^A END Produce comments only in the resulting style file
% \iffalse ^^A BEGIN Ensure the macrocode to the resulting style file
%<*caa> 
% \fi
%    \begin{macrocode}
\newcommand{\incorrect}[1]{#1}
%    \end{macrocode}
% \iffalse ^^A END Ensure the macrocode to the resulting style file
%</caa> 
% \fi
% \end{macro}
% ^^A END \incorrect macro
%
% We now redefine the above macros when the |answers| package option
% is enabled.
% \iffalse ^^A BEGIN Produce comments only in the resulting style file
%<caa>
%<caa>% Declaration of the *answers* option.
% \fi ^^A END Produce comments only in the resulting style file
% \iffalse ^^A BEGIN Ensure the macrocode to the resulting style file
%<*caa> 
% \fi
%    \begin{macrocode}
\DeclareOption{answers}{
  \renewcommand{\explanation}[1]{\textbf{Explanation:}\\#1}
  \renewcommand{\keyed}[1]{\item[\cmark] #1}
  \renewcommand{\distractor}[1]{\item[\xmark] #1}
  \renewcommand{\correct}[1]{\cmark #1}
  \renewcommand{\incorrect}[1]{\xmark #1}
}
%    \end{macrocode}
% \iffalse ^^A END Ensure the macrocode to the resulting style file
%</caa> 
% \fi
%
% \subsubsection{Colouring option}
% \label{sec:colour-option}
% We first declare some internal macros that can be updated when
% accordingly to the option for colouring.
%
% ^^A BEGIN \cmark macro
% \begin{macro}{\cmark}
% \changes{v0.1}{2020/03/29}{Initial version}
% Command |\cmark| is used to indicate the keyed answer or correct
% option in multiple choice question.  When the package options
% |colour| or |color| is enabled, it is also coloured accordingly.
% \iffalse ^^A BEGIN Produce comments only in the resulting style file
%<caa>
%<caa>% Macro to typeset a checked mark using pifont
% \fi ^^A END Produce comments only in the resulting style file
% \iffalse ^^A BEGIN Ensure the macrocode to the resulting style file
%<*caa> 
% \fi
%    \begin{macrocode}
\newcommand{\cmark}{\textcolor{\CAA@cmarkcolour}{\ding{51}}}%
%    \end{macrocode}
% \iffalse ^^A END Ensure the macrocode to the resulting style file
%</caa> 
% \fi
% \end{macro}
% ^^A END \cmark macro
%
% ^^A BEGIN \xmark macro
% \begin{macro}{\xmark}
% \changes{v0.1}{2020/03/29}{Initial version}
% Command |\xmark| is used to indicate the distractor answer or
% incorrect option in multiple choice question.  When the package
% options |colour| or |color| is enabled, it is also coloured
% accordingly.
% \iffalse ^^A BEGIN Produce comments only in the resulting style file
%<caa>
%<caa>% Macro to typeset a crossed mark using pifont
% \fi ^^A END Produce comments only in the resulting style file
% \iffalse ^^A BEGIN Ensure the macrocode to the resulting style file
%<*caa> 
% \fi
%    \begin{macrocode}
\newcommand{\xmark}{\textcolor{\CAA@xmarkcolour}{\ding{55}}}%
%    \end{macrocode}
% \iffalse ^^A END Ensure the macrocode to the resulting style file
%</caa> 
% \fi
% \end{macro}
% ^^A END \xmark macro
%
% ^^A BEGIN \CAA@cmarkcolour macro
% \begin{macro}{\CAA@cmarkcolour}
% Command |\CAA@cmarkcolour| is used as the colour for the checked
% mark.  By default, it is |black|. When the |colour| option is
% enabled, the colour is set to |green|.
% \iffalse ^^A BEGIN Produce comments only in the resulting style file
%<caa>
%<caa>% Macro as the colour for the checked mark. It is *black* by
%<caa>% default, and redefined to be *green* when the package options
%*colour* or *color* are enabled.
% \fi ^^A END Produce comments only in the resulting style file
% \iffalse ^^A BEGIN Ensure the macrocode to the resulting style file
%<*caa> 
% \fi
%    \begin{macrocode}
\newcommand{\CAA@cmarkcolour}{black}
%    \end{macrocode}
% \iffalse ^^A END Ensure the macrocode to the resulting style file
%</caa> 
% \fi
% \end{macro}
% ^^A END \CAA@cmarkcolour macro
%
% ^^A BEGIN \CAA@xmarkcolour macro
% \begin{macro}{\CAA@xmarkcolour}
% Command |\CAA@xmarkcolour| is used as the colour for the crossed
% mark.  By default, it is |black|. When the |colour| option is
% enabled, the colour is set to |red|.
% \iffalse ^^A BEGIN Produce comments only in the resulting style file
%<caa>
%<caa>% Macro as the colour for the crossed mark. It is *black* by
%<caa>% default, and redefined to be *red* when the package options
%*colour* or *color* are enabled.
% \fi ^^A END Produce comments only in the resulting style file
% \iffalse ^^A BEGIN Ensure the macrocode to the resulting style file
%<*caa> 
% \fi
%    \begin{macrocode}
\newcommand{\CAA@xmarkcolour}{black}
%    \end{macrocode}
% \iffalse ^^A END Ensure the macrocode to the resulting style file
%</caa> 
% \fi
% \end{macro}
% ^^A END \CAA@xmarkcolour macro
%
% ^^A BEGIN \setcmarkcolour macro
% \begin{macro}{\setcmarkcolour}
% \changes{v0.1}{2020/03/29}{Initial version}
% Command |\setcmarkcolour| is used to redefine the colour for the
% checked mark.
% \iffalse ^^A BEGIN Produce comments only in the resulting style file
%<caa>
%<caa>% Macro to redefine the colour for checked mark.
%<caa>%
%<caa>% Arguments:
%<caa>% 1. The new colour for the checked mark.
% \fi ^^A END Produce comments only in the resulting style file
% \iffalse ^^A BEGIN Ensure the macrocode to the resulting style file
%<*caa> 
% \fi
%    \begin{macrocode}
\newcommand{\setcmarkcolour}[1]{%
  \renewcommand{\CAA@cmarkcolour}{#1}%
}%
%    \end{macrocode}
% \iffalse ^^A END Ensure the macrocode to the resulting style file
%</caa> 
% \fi
% \end{macro}
% ^^A END \setcmarkcolour macro
%
% ^^A BEGIN \setxmarkcolour macro
% \begin{macro}{\setxmarkcolour}
% \changes{v0.1}{2020/03/29}{Initial version}
% Command |\setxmarkcolour| is used to redefine the colour for the
% crossed mark.
% \iffalse ^^A BEGIN Produce comments only in the resulting style file
%<caa>
%<caa>% Macro to redefine the colour for crossed mark.
%<caa>%
%<caa>% Arguments:
%<caa>% 1. The new colour for the crossed mark.
% \fi ^^A END Produce comments only in the resulting style file
% \iffalse ^^A BEGIN Ensure the macrocode to the resulting style file
%<*caa> 
% \fi
%    \begin{macrocode}
\newcommand{\setxmarkcolour}[1]{%
  \renewcommand{\CAA@xmarkcolour}{#1}%
}%
%    \end{macrocode}
% \iffalse ^^A END Ensure the macrocode to the resulting style file
%</caa> 
% \fi
% \end{macro}
% ^^A END \setxmarkcolour macro
% We now define the |colour| option and set the different colours
% accordingly.
% \iffalse ^^A BEGIN Produce comments only in the resulting style file
%<caa>
%<caa>% Declaration of the *colour* option.
% \fi ^^A END Produce comments only in the resulting style file
% \iffalse ^^A BEGIN Ensure the macrocode to the resulting style file
%<*caa> 
% \fi
%    \begin{macrocode}
\DeclareOption{colour}{
  \setcmarkcolour{green}%
  \setxmarkcolour{red}%
}
%    \end{macrocode}
% \iffalse ^^A END Ensure the macrocode to the resulting style file
%</caa> 
% \fi
% Additionally, we define the |color| package option as an alias of
% |colour|.
% \iffalse ^^A BEGIN Produce comments only in the resulting style file
%<caa>
%<caa>% Declaration of the *color* option as an alias of *colour*.
% \fi ^^A END Produce comments only in the resulting style file
% \iffalse ^^A BEGIN Ensure the macrocode to the resulting style file
%<*caa> 
% \fi
%    \begin{macrocode}
\DeclareOption{color}{
  \ExecuteOptions{colour}
}
%    \end{macrocode}
% \iffalse ^^A END Ensure the macrocode to the resulting style file
%</caa> 
% \fi
% \iffalse ^^A BEGIN Produce comments only in the resulting style file
%<caa>%%%%% END Declaration of options %%%%%
% \fi ^^A END Produce comments only in the resulting style file
%
% \subsubsection{Execution of options}
% \label{sec:options-execution}
% We are now execute the options by executing |\ProcessOptions|
% command.
% \iffalse ^^A BEGIN Produce comments only in the resulting style file
%<caa>
%<caa>%%%%% BEGIN Execution of options %%%%%
%<caa>% ========================
% \fi ^^A END Produce comments only in the resulting style file
% \iffalse ^^A BEGIN Ensure the macrocode to the resulting style file
%<*caa> 
% \fi
%    \begin{macrocode}
\ProcessOptions
%    \end{macrocode}
% \iffalse ^^A END Ensure the macrocode to the resulting style file
%</caa> 
% \fi
% \iffalse ^^A BEGIN Produce comments only in the resulting style file
%<caa>%%%%% END Execution of options %%%%%
% \fi ^^A END Produce comments only in the resulting style file
%
% \subsection{Typesetting Questions}
% \label{sec:types-comp-aided}
%
% \iffalse ^^A BEGIN Produce comments only in the resulting style file
%<caa>%%%%% BEGIN Typesetting Questions %%%%%
% \fi ^^A END Produce comments only in the resulting style file
%
% We define macros for typesetting questions for computer-aided
% assessments.
%
% \subsubsection{Question Categories}
% \label{sec:question-categories}
%
% ^^A BEGIN \newcategory macro
% \begin{macro}{\newcategory}
% \changes{v0.1}{2020/03/29}{Initial version}
% We group questions into categories and we track the number of
% questions and the marks allocated in each category. As a result, we
% create two counters for each newly created category.
% \iffalse ^^A BEGIN Produce comments only in the resulting style file
%<caa>
%<caa>% Macro to create a new category. Two counters are created to
%<caa>% track the number of questions and the marks for each
%<caa>% category. 
%<caa>%
%<caa>% Arguments:
%<caa>% 1. The name of the new category.
% \fi ^^A END Produce comments only in the resulting style file
% \iffalse ^^A BEGIN Ensure the macrocode to the resulting style file
%<*caa> 
% \fi
%    \begin{macrocode}
\newcommand{\newcategory}[1]{%
  \newtotcounter{#1}
  \setcounter{#1}{0}
  \newtotcounter{#1mark}
  \setcounter{#1mark}{0}
}
%    \end{macrocode}
% \iffalse ^^A END Ensure the macrocode to the resulting style file
%</caa> 
% \fi
% \end{macro}
% ^^A END \newcategory macro
%
% We create some default categories |Easy| for easy difficulty
% questions, |Medium| for medium difficulty questions, |Hard|
% for hard difficulty questions, and |Total| for the all
% questions.
% \iffalse ^^A BEGIN Produce comments only in the resulting style file
%<caa>
%<caa>% Create default categories: Easy, Medium, Hard, and Total.
% \fi ^^A END Produce comments only in the resulting style file
% ^^A BEGIN Easy category
% \begin{macro}{Easy category}
% \changes{v0.1}{2020/03/29}{Initial version}
% A category for easy questions.
% \iffalse ^^A BEGIN Ensure the macrocode to the resulting style file
%<*caa>
% \fi
%    \begin{macrocode}
\newcategory{Easy}
%    \end{macrocode}
% \iffalse ^^A END Ensure the macrocode to the resulting style file
%</caa> 
% \fi
% \end{macro}
% ^^A END Easy category
% ^^A BEGIN Medium category
% \begin{macro}{Medium category}
% \changes{v0.1}{2020/03/29}{Initial version}
% A category for medium questions.
% \iffalse ^^A BEGIN Ensure the macrocode to the resulting style file
%<*caa>
% \fi
%    \begin{macrocode}
\newcategory{Medium}
%    \end{macrocode}
% \iffalse ^^A END Ensure the macrocode to the resulting style file
%</caa> 
% \fi
% \end{macro}
% ^^A END Medium category
% ^^A BEGIN Hard category
% \begin{macro}{Hard category}
% \changes{v0.1}{2020/03/29}{Initial version}
% A category for hard questions.
% \iffalse ^^A BEGIN Ensure the macrocode to the resulting style file
%<*caa>
% \fi
%    \begin{macrocode}
\newcategory{Hard}
%    \end{macrocode}
% \iffalse ^^A END Ensure the macrocode to the resulting style file
%</caa> 
% \fi
% \end{macro}
% ^^A END Hard category
% ^^A BEGIN Total category
% \begin{macro}{Total category}
% \changes{v0.1}{2020/03/29}{Initial version}
% A category for all questions.
% \iffalse ^^A BEGIN Ensure the macrocode to the resulting style file
%<*caa>
% \fi
%    \begin{macrocode}
\newcategory{Total}
%    \end{macrocode}
% \iffalse ^^A END Ensure the macrocode to the resulting style file
%</caa>
% \fi
% \end{macro}
% ^^A END Total category
%
% ^^A BEGIN \addtocategory macro
% \begin{macro}{\addtocategory}
% \changes{v0.1}{2020/03/29}{Initial version}
% An utiltiy macro is created to add a question and its marks to the
% category.
% \iffalse ^^A BEGIN Produce comments only in the resulting style file
%<caa>
%<caa>% Macro to create a question and its marks to a given category.
%<caa>%
%<caa>% Arguments:
%<caa>% 1. The name of the category.
%<caa>% 2. The number of marks to be added to the category.
% \fi ^^A END Produce comments only in the resulting style file
% \iffalse ^^A BEGIN Ensure the macrocode to the resulting style file
%<*caa> 
% \fi
%    \begin{macrocode}
\newcommand{\addtocategory}[2]{%
  \refstepcounter{#1}
  \addtocounter{#1mark}{#2}
}
%    \end{macrocode}
% \iffalse ^^A END Ensure the macrocode to the resulting style file
%</caa> 
% \fi
% \end{macro}
% ^^A END \addtocategory macro
%
% \subsubsection{Create Questions}
% \label{sec:create-questions}
%
% ^^A BEGIN \newquestion macro
% \begin{macro}{\newquestion}
% This macro is to create a question. Each question has a label,
% marks, category, and content. The question and its marks are added
% to the given category.  The total number of questions and marks are
% increased accordingly.  The nomenclature is updated with the
% question and its marks.
% \iffalse ^^A BEGIN Produce comments only in the resulting style file
%<caa>
%<caa>% Macro to create a new question. The question and its marks are
%<caa>% added to the given category. The total number of questions and
%<caa>% marks are increased accordingly. The list nomenclature is
%<caa>% updated with the question and its mark.
%<caa>%
%<caa>% Arguments:
%<caa>% 1. (Optional) Question category (default is Easy) 
%<caa>% 2. Question label
%<caa>% 3. Marks associated with the question
%<caa>% 4. Question content.
% \fi ^^A END Produce comments only in the resulting style file
% \iffalse ^^A BEGIN Ensure the macrocode to the resulting style file
%<*caa>
% \fi
%    \begin{macrocode}
\newcommand{\newquestion}[4][Easy]{%
  \listadd{\CAA@list}{#2}%
  \addtocategory{#1}{#3}
  \addtocategory{Total}{#3}
  \ifthenelse{#3 = 1}{%
    \nomenclature{#2 (#1)~~~~~}{#3 mark}%
  }{%
    \nomenclature{#2 (#1)~~~~~}{#3 marks}%
  }%
  
  \noindent\textbf{#2 (#1)} [\CAA@plural{#3}{mark}]\\%
  #4%
}%
%    \end{macrocode}
% \iffalse ^^A END Ensure the macrocode to the resulting style file
%</caa>
% \fi
% \end{macro}
% ^^A END \newquestion macro
%
% \subsubsection{Question Summary}
% \label{sec:question-summary}
%
% We first create some utility macros for produce the question
% summary, e.g., the number of questions or marks in a certain
% category.
%
% ^^A BEGIN \CAA@ofstrequal macro
% \begin{macro}{\CAA@ifstrequal}
% This macro provides a wrapper for |\ifstrequal| to ensure that the
% first argument is properly expanded.
% \iffalse ^^A BEGIN Produce comments only in the resulting style file
%<caa>
%<caa>% A wrapper for \ifstrequal to make sure that the first argument is
%<caa>% properly expanded.
% \fi ^^A END Produce comments only in the resulting style file
% \iffalse ^^A BEGIN Ensure the macrocode to the resulting style file
%<*caa>
% \fi
%    \begin{macrocode}
\newcommand{\CAA@ifstrequal}{\expandafter\ifstrequal\expandafter}
%    \end{macrocode}
% \iffalse ^^A END Ensure the macrocode to the resulting style file
%</caa>
% \fi
% \end{macro}
% ^^A END \CAA@ifstrequal macro
%
% ^^A BEGIN \CAA@plural macro
% \begin{macro}{\CAA@plural}
% This macro is to print the number with singular/plural noun.
% \iffalse ^^A BEGIN Produce comments only in the resulting style file
%<caa>
%<caa>% This macro is to print the number of item with plural
%<caa>% suffix. If the input number is 1, the singular noun is
%<caa>% used. Otherwise the plural noun is used.
%<caa>%
%<caa>% Arguments:
%<caa>% 1. (Optional) The plural noun.
%<caa>% 2. The number.
%<caa>% 3. The singular noun.
% \fi ^^A END Produce comments only in the resulting style file
% \iffalse ^^A BEGIN Ensure the macrocode to the resulting style file
%<*caa>
% \fi
%    \begin{macrocode}
\newcommand{\CAA@plural}[3][]{%
  \CAA@ifstrequal{#1}{}{%
    \ifthenelse{#2 = 1}{#2 #3}{#2 #3s}%
  }{%
    \ifthenelse{#2 = 1}{#2 #3}{#2 #1}%
  }%
}%
%    \end{macrocode}
% \iffalse ^^A END Ensure the macrocode to the resulting style file
%</caa>
% \fi
% \end{macro}
% ^^A END \CAA@plural macro
%
% ^^A BEGIN \CAA@pluralref macro
% \begin{macro}{\CAA@pluralref}
% This macro is to print the number with singular/plural noun.
% \iffalse ^^A BEGIN Produce comments only in the resulting style file
%<caa>
%<caa>% This macro is to print the number getting from the reference
%<caa>% with singular/plural noun. If the input number (getting from
%<caa>% the reference) is 1, the singular noun is used. Otherwise the
%<caa>% plural noun is used.
%<caa>%
%<caa>% Arguments:
%<caa>% 1. (Optional) The plural noun.
%<caa>% 2. The reference that stored the number.
%<caa>% 3. The singular noun.
% \fi ^^A END Produce comments only in the resulting style file
% \iffalse ^^A BEGIN Ensure the macrocode to the resulting style file
%<*caa>
% \fi
%    \begin{macrocode}
\newcommand{\CAA@pluralref}[3][]{%
  \CAA@ifstrequal{#1}{}{%
    \ifthenelse{\totvalue{#2} = 1}{\total{#2} #3}{\total{#2} #3s}%
  }{%
    \ifthenelse{\totvalue{#2} = 1}{\total{#2} #3}{\total{#2} #1}%
  }%
}%
%    \end{macrocode}
% \iffalse ^^A END Ensure the macrocode to the resulting style file
%</caa>
% \fi
% \end{macro}
% ^^A END \CAA@pluralref macro
%
% ^^A BEGIN \printcategory macro
% \begin{macro}{\printcategory}
% This macro print the information about a category (i.e., the number
% of questions and the number of marks.
% \iffalse ^^A BEGIN Produce comments only in the resulting style file
%<caa>
%<caa>% This macro print the information about a category (i.e., the
%<caa>% number of questions and the number of marks. 
% \fi ^^A END Produce comments only in the resulting style file
% \iffalse ^^A BEGIN Ensure the macrocode to the resulting style file
%<*caa>
% \fi
%    \begin{macrocode}
\newcommand{\printcategory}[1]{%
  \CAA@pluralref{#1}{question} (\CAA@pluralref{#1mark}{mark})
}%
%    \end{macrocode}
% \iffalse ^^A END Ensure the macrocode to the resulting style file
%</caa>
% \fi
% \end{macro}
% ^^A END \CAA@pluralref macro
%
% \iffalse ^^A BEGIN Produce comments only in the resulting style file
%<caa>%%%%% END Typesetting Questions %%%%%
% \fi ^^A END Produce comments only in the resulting style file
%
% \Finale
\endinput